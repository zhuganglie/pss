\documentclass[aspectratio=169]{ctexbeamer}
\usepackage{fontspec}
\usepackage{microtype}
\usepackage{graphicx}
\usepackage{booktabs}
\usepackage{xcolor}
\usepackage{hyperref}
\usepackage{listings}

% Theme
\usetheme{Madrid}

% Title, Author, etc.
\title{威权主义下的权力分享}
\subtitle{概念、机制与策略}
\author{\small Anne Meng, Jack Paine, and Robert Powell \\ (Annual Review of Political Science, 2023)}
\date{\today}
\institute{摘要与翻译}

\begin{document}

% Title Frame
\begin{frame}
  \titlepage
\end{frame}

% Agenda
\begin{frame}{议程}
  \tableofcontents
\end{frame}

\section{摘要 (Abstract)}
\begin{frame}{摘要}
  \begin{itemize}
    \item 为研究威权政体中的权力分享提供一个统一的语言。
    \item 权力分享协议不仅涉及战利品分享,还包括建立一个执行机制。
    \item 如果没有权力的重新分配使统治者难以反悔,那么协议就不能真正分享权力。
    \item 当制度薄弱时,如果挑战者拥有强制手段来保卫其战利品,自我执行的权力分享仍然是可能的。
    \item 然而,这把双刃剑也可能让挑战者推翻统治者。
  \end{itemize}
\end{frame}

\section{导言 (Introduction)}
\begin{frame}{导言}
  \begin{itemize}
    \item 没有独裁者是天生稳固的。他们面临来自政权精英、反对派团体和其他社会行动者的威胁。
    \item 近期研究的核心观点是,统治者必须分享权力以获得其核心圈子之外的行动者的支持。
    \item 本文旨在为研究威权主义权力分享提供一个统一的语言。
  \end{itemize}
\end{frame}

\section{权力分享的概念化 (Conceptualizing Power Sharing)}
\begin{frame}{权力分享的概念化}
  一个权力分享协议必须满足两个不同的要求:
  \begin{enumerate}
    \item<1-> 在各方之间分享战利品。
    \item<2-> 以一种使统治者难以反悔的方式重新分配权力。
  \end{enumerate}
  \vspace{1cm}
  \uncover<3->{缺乏可信执行机制的纯粹战利品转移不构成权力分享。}
\end{frame}

\section{执行机制 (Enforcement Mechanisms)}
\subsection{制度性执行 (Institutional Enforcement)}
\begin{frame}{制度性执行}
  \begin{itemize}
    \item 制度性让步(如授权议程控制、赋权第三方执行者)可以通过增加统治者反悔的成本来重新分配权力。
    \item \textbf{弱制度的困境 (Catch-22 of Weak Institutions)}: 制度的薄弱恰恰是承诺不可信的原因。
    \item \textbf{强制度的雪球效应 (Snowball Effect of Strong Institutions)}: 最初小小的制度让步可能会像滚雪球一样,最终导致统治者失去比预期更多的权力。
  \end{itemize}
\end{frame}

\subsection{强制性执行 (Coercive Enforcement)}
\begin{frame}{强制性执行}
  \begin{itemize}
    \item 当制度薄弱时,如果挑战者拥有保卫其战利品的强制手段,自我执行的权力分享是可能的。
    \item 统治者可以直接分享强制手段的控制权,例如:
    \begin{itemize}
      \item 允许对手控制军队中的高级职位。
      \item 允许叛乱团体保留其武装。
    \end{itemize}
    \item 这会产生\textbf{威胁增强效应 (Threat-Enhancing Effect)},挑战者可能利用这些能力发动进攻并推翻统治者。
  \end{itemize}
\end{frame}

\section{权力分享的策略 (Strategies of Sharing Power)}
\begin{frame}{权力分享的策略}
  促使战略性统治者分享权力的三个条件:
  \begin{enumerate}
    \item<1-> \textbf{挑战者的可信度}: 挑战者必须能够可信地惩罚不分享权力的统治者。
    \item<2-> \textbf{挑战者的意愿}: 如果统治者分享权力,挑战者必须愿意放弃采取有害行动。
    \item<3-> \textbf{统治者的意愿}: 统治者必须愿意接受权力分享协议所带来的约束和租金损失。
  \end{enumerate}
\end{frame}

\section{结论 (Conclusion)}
\begin{frame}{总结与未来研究方向}
  \begin{itemize}
    \item 本文通过三个主要贡献,对关于威权主义权力分享的文献进行了梳理和重组。
    \item \textbf{第一},将权力分享概念化为同时满足“分享战利品”和“重新分配权力”两个要求。
    \item \textbf{第二},识别了制度性和强制性两种主要的执行机制。
    \item \textbf{第三},阐述了促进权力分享的三个条件:挑战者可信度、挑战者意愿和统治者意愿。
    \item 未来的研究需要更深入地审视这些执行机制。
  \end{itemize}
\end{frame}

\begin{frame}
  \begin{center}
    \Huge 谢谢!\\
    \vspace{1cm}
    \Large 问答环节
  \end{center}
\end{frame}

\end{document}