\chapter{附录}

\section{术语解释 (Glossary)}

本部分收录了本书中提及的核心概念和术语,旨在帮助读者更好地理解民粹主义及其相关现象。

\begin{itemize}
    \item \textbf{民粹主义 (Populism)}:一种政治逻辑或政治风格,其核心是将社会想象成两个同质化且相互对抗的阵营:“纯洁的、统一的人民”与“腐败的、自私的精英”之间的道德斗争。民粹主义者将自己定位为“人民”的唯一真实代表。

    \item \textbf{纯洁的、统一的人民 (The Pure, Unified People)}:民粹主义叙事中被描绘为勤劳善良、拥有常识、代表国家真正意志和核心价值的沉默大多数。

    \item \textbf{腐败的、自私的精英 (The Corrupt, Selfish Elite)}:民粹主义叙事中被指责为脱离群众、道德败坏、滥用权力、只顾自身利益,甚至与外国势力勾结的群体。

    \item \textbf{经济民粹主义 (Economic Populism)}:民粹主义与左翼议程结合时的一种表现形式,通常将“人民”定义为经济弱势群体,将“精英”定义为大资本家、金融寡头等,强调财富再分配和经济正义。

    \item \textbf{民族主义民粹主义 (Nationalist Populism)}:民粹主义与右翼议程结合时的一种表现形式,通常将“人民”定义为特定民族、种族或文化群体的“土著”成员,将“精英”定义为推行多元文化主义、纵容移民的“世界主义者”,强调国家主权和传统文化。

    \item \textbf{文化民粹主义 (Cultural Populism)}:与民族主义民粹主义类似,侧重于捍卫传统文化和生活方式,对抗被视为外来或“自由派精英”强加的文化变迁。

    \item \textbf{全球化 (Globalization)}:商品、服务、资本、技术和信息在全球范围内自由流动的过程,本书探讨了其带来的经济和文化后果,以及如何成为民粹主义兴起的背景。

    \item \textbf{历史的终结 (End of History)}:弗朗西斯·福山提出的理论,认为人类历史的意识形态之争已结束,自由民主制与市场经济取得了最终胜利。本书认为民粹主义的兴起反驳了这一论断。

    \item \textbf{铁锈地带 (Rust Belt)}:美国中西部和东北部地区,曾是传统工业中心,因去工业化而衰落,成为民粹主义的重要票仓。

    \item \textbf{去工业化 (Deindustrialization)}:发达国家传统制造业衰退,生产外迁至劳动力成本更低地区的过程,导致大量蓝领工人失业。

    \item \textbf{自动化焦虑 (Automation Anxiety)}:技术进步(如机器人、人工智能)取代重复性劳动岗位,导致工人对失业和技能贬值的担忧。

    \item \textbf{中产阶级困境 (Middle-class Predicament)}:中产阶级实际收入停滞不前,生活成本(如房价、教育、医疗)上涨,向上流动阶梯受阻的普遍压力。

    \item \textbf{2008年金融危机 (2008 Financial Crisis)}:源自美国次贷危机的全球性金融危机,被视为彻底击垮普通民众对现有经济和政治体制信任的关键事件。

    \item \textbf{大置换 (Le Grand Remplacement)}:法国极右翼作家雷诺·加缪提出的阴谋论,声称欧洲白人基督徒人口正被非欧洲移民系统性“置换”。

    \item \textbf{政治正确 (Political Correctness)}:指在言行上避免冒犯或歧视特定群体,民粹主义者常将其视为“自由派精英”压制言论自由的工具。

    \item \textbf{想象的共同体 (Imagined Community)}:本尼迪克特·安德森提出的概念,指民族是人们在心理上构建的、共享身份认同的群体,民粹主义常利用对“黄金时代”的怀旧来强化这种想象。

    \item \textbf{故土的陌生人 (Stranger in Their Own Land)}:社会学家阿莉·霍赫希尔德提出的概念,指一些群体(如美国白人蓝领)感觉在自己的国家中,因人口结构、社会规范变化而变得边缘化、不被理解。

    \item \textbf{第三条道路 (Third Way)}:20世纪末西方中左翼政党(如英国工党、美国民主党)试图在传统左右翼之间找到平衡的政治路线,拥抱全球化和市场经济,同时强调社会公平。

    \item \textbf{政治阶层 (Political Class)}:指政治职业化后,形成的一个与普通民众生活渐行渐远、自我封闭的精英群体。

    \item \textbf{旋转门 (Revolving Door)}:政府官员卸任后迅速进入其曾监管的大公司或金融机构担任高薪职位的现象,常引发利益冲突质疑。

    \item \textbf{净手运动 (Mani Pulite)}:20世纪90年代初意大利大规模反腐调查,彻底摧毁了该国战后建立的政党体系。

    \item \textbf{五星运动 (MoVimento 5 Stelle)}:意大利一个由喜剧演员创立的激进民粹主义政党,以“反政治”和“把他们都送回家”为口号。

    \item \textbf{洗车行动 (Operação Lava Jato)}:巴西大规模反腐调查,揭露了巴西国家石油公司(Petrobras)的腐败网络,为博索纳罗上台铺平道路。

    \item \textbf{非自由民主 (Illiberal Democracy)}:一种看似保留民主形式(如选举),但实质上侵蚀民主内核(如司法独立、媒体自由、公民社会)的统治模式,以匈牙利欧尔班政府为典型。

    \textbf{系统性“捕获”国家机器 (Systemic Capture of State Institutions)}:指民粹主义政府通过修改法律、安插亲信等手段,逐步控制和削弱国家关键机构(如法院、媒体、选举委员会)的独立性。

    \item \textbf{主导文化 (Leitkultur)}:德国选择党(AfD)提出的概念,指以基督教传统、德语、德国历史和特定社会规范为基础的文化共同体,用于排斥外来文化。

    \item \textbf{禁毒战争 (War on Drugs)}:菲律宾总统杜特尔特发起的严厉打击毒品犯罪的行动,伴随着大量法外处决,体现了强人救世主模式的铁腕手段。

    \item \textbf{印度教特性 (Hindutva)}:印度教民族主义意识形态,主张印度是一个印度教国家,莫迪政府以此为基础推行政策。

    \item \textbf{卖茶人 (Chaiwala)}:印度总理莫迪的自称,强调其出身普通,与精英形成对比,以赢得底层民众认同。

    \item \textbf{政策瘫痪 (Policy Paralysis)}:政府因腐败、官僚主义或政治僵局而无法有效制定和执行政策的状态。

    \item \textbf{玻利瓦尔革命 (Bolivarian Revolution)}:委内瑞拉总统查韦斯发起的政治运动,旨在实现“21世纪社会主义”,强调反帝国主义和财富再分配。

    \item \textbf{玻利瓦尔任务 (Misiones Bolivarianas)}:查韦斯政府利用石油收入推行的大规模社会福利计划,如免费医疗、扫盲等。

    \item \textbf{21世纪社会主义 (21st Century Socialism)}:查韦斯提出的政治经济模式,旨在超越传统社会主义,通过国家控制关键资源和大规模社会福利实现社会公平。

    \item \textbf{第四次转型 (La Cuarta Transformación)}:墨西哥总统洛佩斯·奥夫拉多尔提出的执政理念,旨在根除腐败、重塑国家,被视为左翼民粹主义的2.0版。

    \item \textbf{权力黑手党 (La Mafia del Poder)}:洛佩斯·奥夫拉多尔用来指代墨西哥腐败的传统政治和经济精英的术语。

    \item \textbf{假新闻 (Fake News)}:民粹主义领袖用来攻击任何对其不利报道的标签,旨在摧毁主流媒体的公信力。

    \item \textbf{人民的敌人 (Enemy of the People)}:民粹主义领袖用来指责批评者(尤其是媒体)的煽动性词语,将政治分歧道德化为正邪之战。

    \item \textbf{深层政府 (Deep State)}:阴谋论中指称的、由非民选官员(如情报人员、官僚)组成的秘密网络,被认为在幕后操纵政府,阻碍“人民意志”。

    \item \textbf{匿名者Q (QAnon)}:美国极右翼阴谋论,声称世界被一个由信奉撒旦的恋童癖组成的“深层政府”阴谋集团所统治,特朗普是来拯救世界的救世主。

    \item \textbf{信息茧房 (Information Cocoon / Filter Bubble)}:指人们在网络上因算法推荐而只接触到与自己观点一致的信息,导致视野狭窄、难以接触不同观点。

    \item \textbf{认知失调 (Cognitive Dissonance)}:当个体持有相互矛盾的信念、态度或行为时,所体验到的心理不适感,常导致个体拒绝与自身观点相悖的信息。

    \item \textbf{动机性推理 (Motivated Reasoning)}:个体在处理信息时,倾向于以支持自己已有信念或偏见的方式进行推理,而非客观分析。

    \item \textbf{政治极化 (Political Polarization)}:社会成员在政治观点上的分歧日益扩大,中间地带萎缩,不同政治阵营之间敌对情绪加剧的现象。

    \item \textbf{回音室效应 (Echo Chamber Effect)}:与信息茧房类似,指人们在社交媒体等环境中,只听到与自己观点相似的声音,强化原有信念,难以接触不同意见。

    \item \textbf{本国优先 (Our Country First / America First)}:民粹主义内向的国际政策观,将狭隘的国家利益置于一切之上,对国际合作持怀疑甚至敌视态度。

    \item \textbf{多边主义 (Multilateralism)}:通过多方协商、共同制定规则、集体行动来解决全球性问题的国际合作原则。

    \item \textbf{卫生隔离带 (Cordon Sanitaire)}:主流政党之间达成共识,拒绝与被视为极端的民粹主义政党进行任何形式的官方合作的策略。

    \item \textbf{防火墙 (Brandmauer)}:德国主流政党用来形容拒绝与极右翼“德国选择党”(AfD)合作的“卫生隔离带”策略。

    \item \textbf{议题吸纳 (Issue Co-optation)}:主流政党主动采纳一部分民粹主义政党的核心主张,以争取其支持者的策略。

    \item \textbf{另类愿景 (Alternative Vision)}:主流政治力量提出一个全新的、积极的、面向未来的替代愿景,以对抗民粹主义者那套基于怀旧、愤怒和恐惧的叙事。

    \item \textbf{草根社会运动 (Grassroots Social Movements)}:由普通公民自发组织、自下而上发起的社会行动,常用于对抗民粹主义政府的政策或价值观。

    \item \textbf{女性大游行 (Women's March)}:2017年在美国及全球范围内发起的,抗议特朗普性别歧视言论和排外政策的大规模游行。

    \item \textbf{黑人的命也是命 (Black Lives Matter, BLM)}:源于美国对警察暴力和系统性种族歧视的抗议,后演变为全球性社会运动。

    \item \textbf{气候罢课 (Fridays for Future)}:由青年学生发起的全球性气候行动,要求政府采取紧急气候行动。

    \item \textbf{反抗灭绝 (Extinction Rebellion)}:采取非暴力不合作策略,旨在迫使政府正视气候危机的环保运动。

    \item \textbf{调查报道 (Investigative Journalism)}:记者通过深入挖掘、揭露权力滥用和不法行为的新闻报道形式,是守护真相的重要力量。

    \item \textbf{事实核查 (Fact-checking)}:对政治人物言论、社交媒体谣言等进行核查,辨别真伪,对抗虚假信息的活动。

    \item \textbf{司法制衡 (Judicial Checks and Balances)}:独立的司法系统对政府权力进行监督和限制,是民主制度的重要组成部分。

    \item \textbf{选举反弹 (Electoral Backlash)}:选民通过投票拒绝或推翻民粹主义统治的现象。

    \item \textbf{官僚体系的抵抗 (Bureaucratic Resistance)}:专业的公务员和技术官僚在面对政治压力时,坚守职业道德和专业精神,抵制政治化的行为。

    \item \textbf{排他性的民粹主义 (Exclusive Populism)}:将“人民”定义为同质化的、排他的群体,通过树立“内部敌人”和“外部威胁”来巩固团结的民粹主义形式。

    \item \textbf{包容性的民主 (Inclusive Democracy)}:一种复兴的民主模式,承认并回应真实的不满,用多元、开放、基于共同公民身份和普世价值的“我们”来对抗排他性,并捍卫和革新民主制度。
\end{itemize}

\section{建议阅读书目 (Suggested Readings)}

本书在撰写过程中参考了大量关于民粹主义、政治学、社会学和国际关系领域的著作与研究。以下列出一些与本书主题紧密相关、有助于读者进一步深入了解的经典及当代作品。请注意,本书正文并未采用严格的学术引用格式,此列表旨在提供延伸阅读的指引。

\begin{itemize}
    \item \textbf{关于民粹主义的理论与概念:}
    \begin{itemize}
        \item Mudde, Cas, and Cristóbal Rovira Kaltwasser. \textit{Populism: A Very Short Introduction}. Oxford University Press, 2017. (民粹主义的入门读物,清晰界定概念)
        \item Jan-Werner Müller. \textit{What Is Populism?} University of Pennsylvania Press, 2016. (深入探讨民粹主义的道德主张和反多元主义倾向)
        \item Laclau, Ernesto. \textit{On Populist Reason}. Verso, 2005. (从后结构主义角度分析民粹主义的理论基础)
    \end{itemize}

    \item \textbf{关于全球化与经济不满:}
    \begin{itemize}
        \item Rodrik, Dani. \textit{The Globalization Paradox: Democracy and the Future of the World Economy}. W. W. Norton \& Company, 2011. (探讨全球化、民主和国家主权之间的三难困境)
        \item Milanovic, Branko. \textit{Global Inequality: A New Approach for the Age of Globalization}. Harvard University Press, 2016. (分析全球收入不平等及其对政治的影响)
    \end{itemize}

    \item \textbf{关于身份与文化焦虑:}
    \begin{itemize}
        \item Hochschild, Arlie Russell. \textit{Strangers in Their Own Land: Anger and Mourning on the American Right}. The New Press, 2016. (对美国茶党支持者深入的民族志研究,理解文化和情感驱动的政治)
        \item Fukuyama, Francis. \textit{Identity: The Demand for Dignity and the Politics of Resentment}. Farrar, Straus and Giroux, 2018. (探讨身份政治的兴起及其对自由民主的挑战)
    \end{itemize}

    \item \textbf{关于主流政治失灵:}
    \begin{itemize}
        \item Mounk, Yascha. \textit{The People vs. Democracy: Why Our Freedom Is in Danger and How to Save It}. Harvard University Press, 2018. (分析自由民主的“脱钩”现象,即自由与民主原则的分离)
        \item Levitsky, Steven, and Daniel Ziblatt. \textit{How Democracies Die}. Crown, 2018. (探讨民主如何通过看似合法的手段被侵蚀,而非军事政变)
    \end{itemize}

    \item \textbf{关于特定国家案例:}
    \begin{itemize}
        \item Vance, J. D. \textit{Hillbilly Elegy: A Memoir of a Family and Culture in Crisis}. Harper, 2016. (从个人视角描绘美国锈带地区的社会经济困境)
        \item Ressa, Maria. \textit{How to Stand Up to a Dictator: The Fight for Our Future}. Harper, 2022. (菲律宾记者玛丽亚·雷萨对抗杜特尔特政府的亲身经历)
        \item Snyder, Timothy. \textit{The Road to Unfreedom: Russia, Europe, America}. Tim Duggan Books, 2018. (探讨威权主义和民粹主义在欧洲和美国的影响)
    \end{itemize}

    \item \textbf{关于“后真相”与信息环境:}
    \begin{itemize}
        \item McIntyre, Lee. \textit{Post-Truth}. The MIT Press, 2018. (分析“后真相”现象及其对民主的威胁)
        \item Pariser, Eli. \textit{The Filter Bubble: What the Internet Is Hiding from You}. Penguin Press, 2011. (解释“信息茧房”的形成及其对信息获取的影响)
    \end{itemize}

    \item \textbf{关于民主的韧性与应对:}
    \begin{itemize}
        \item Putnam, Robert D. \textit{Bowling Alone: The Collapse and Revival of American Community}. Simon \& Schuster, 2000. (探讨社会资本的衰退及其对民主的影响)
        \item Diamond, Larry. \textit{Ill Winds: Saving Democracy from Russian Rage, Chinese Ambition, and American Complacency}. Penguin Press, 2019. (分析全球民主面临的挑战及应对策略)
    \end{itemize}
\end{itemize}