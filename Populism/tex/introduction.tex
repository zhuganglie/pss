\chapter{\textbf{引言:群众的怒吼}}

2021年1月6日,华盛顿特区,寒风凛冽。但比天气更冰冷的,是弥漫在美国政治心脏的惊骇与暴戾。数以千计的民众,挥舞着各式旗帜,高喊着口号,如潮水般冲向国会山——那座平日里象征着美式民主庄严与秩序的殿堂。他们砸碎玻璃,推倒路障,与警察激烈冲突,冲进议事大厅,试图阻止一场他们坚信被“窃取”的总统选举结果的认证。摄像机镜头捕捉到一张张愤怒、迷茫甚至狂喜的面孔:有人头戴牛角帽,身披兽皮,如同原始部落的萨满;有人身着迷彩服,手持束缚带,仿佛城市游击队员;更多的人,则是平日里看似普通的美国公民,此刻却化身为“爱国者”,誓要“夺回”他们的国家。这场“勤王风暴”震惊了世界。电视屏幕前,全球的观众目睹了超级大国民主象征的陷落,建制派政治家们语无伦次地谴责,评论员们则努力寻找恰当的词汇来形容这超现实的一幕——这是暴动?是政变未遂?还是一场失控的政治嘉年华?

而在大洋彼岸,或是在同一片土地的不同角落,类似的场景以不同的面貌上演。在印度,总理纳伦德拉·莫迪的集会现场,往往是数十万乃至上百万人汇聚的海洋。支持者们头戴印有他头像的帽子,挥舞着橙色的旗帜,聆听着他充满力量的演说,眼神中充满了近乎宗教般的虔诚与狂热。他们相信,莫迪是那个能带领印度走向伟大复兴的强人,是那个能涤荡腐败、捍卫印度教特性、让普通人扬眉吐气的领袖。这种原始的、喷薄欲出的集体能量,与那些在达沃斯论坛上讨论全球治理、在联合国会议厅里协商国际规则、在顶级学府里分析数据模型的精英们的冷静、理性与程序化形成了鲜明,甚至有些刺眼的对比。当“群众”以如此粗粝、直接、甚至破坏性的方式表达他们的意志时,那些习惯了在既定框架内运作的建制派精英们,往往只剩下困惑、错愕,以及一种深切的无力感。

这便是我们这个时代最令人困惑的谜题之一,也是本书试图解答的核心问题:在一个全球化联系空前紧密、科技发展日新月异、人类社会似乎正朝着更开放、更理性、更包容方向迈进的时代,为何来自截然不同文化背景、经济发展水平迥异的国度——从西方老牌民主国家到亚洲新兴经济体,从欧洲福利国家到拉丁美洲发展中国家——的数以百万计的民众,会同时将他们的选票、他们的信任、他们的希望,投向那些言辞激烈地抨击“体制”、不遗余力地寻找“替罪羊”、甚至公然挑战自由民主制度根基的领袖?

我们曾被告知,历史已然“终结”于自由民主与市场经济的胜利;我们曾相信,互联网会打破信息壁垒,让沟通更透明,让社会更扁平;我们曾期望,经济全球化会带来共同繁荣,弥合分歧。然而,现实却展现出另一幅图景:社交媒体上充斥着“我们”与“他们”的尖锐对立,阴谋论像病毒一样传播;经济增长的果实分配不均,加剧了社会内部的断裂感;曾经被视为普世价值的民主规范和人权理念,在一些地方正受到前所未有的侵蚀。那些承诺“让国家再次伟大”、“夺回控制权”、“将权力还给人民”的口号,为何具有如此强大的魔力?那些将复杂问题简单化、将特定群体妖魔化、将自身塑造为“人民唯一代言人”的政治人物,为何能够赢得如此广泛的拥趸?他们究竟触动了现代人心灵深处的哪一根弦?是经济的困顿、身份的焦虑、文化的失落,还是对现有政治秩序的彻底绝望?或者,是这一切因素错综复杂地交织在一起,形成了一股难以抗拒的时代暗流?

这股暗流,便是我们通常所说的“民粹主义”。它如同一个幽灵,在全球政治舞台上游荡,时而以左翼的面目出现,时而以右翼的姿态登场;时而表现为温和的抗议,时而演变成激烈的冲突。它让一些人欢欣鼓舞,看到了变革的希望;也让另一些人忧心忡忡,预感到混乱的降临。理解这股力量,剖析其运作的逻辑,洞察其带来的影响,已成为我们这个时代无法回避的课题。

在深入探讨之前,我们必须首先尝试揭开“民粹主义”这个词语的神秘面纱。它几乎是当今政治评论中使用频率最高的词汇之一,但同时也可能是被误解最深、被滥用最多的概念。媒体报道中,它常常被随意地贴在各种不满现状的政治运动或风格激进的政治人物身上,有时甚至被用作一个含义模糊的贬义标签,等同于煽动、非理性或反智。然而,简单地将其污名化,并不能帮助我们理解其产生的深层原因和广泛影响。

首先需要明确的是,民粹主义并非一种像社会主义、自由主义或保守主义那样内容完整、体系严密的意识形态。它没有一套固定的政策纲领,比如对于税收应该多高、福利应该多广、或者国家应该如何干预经济等具体问题,民粹主义者内部可能存在巨大分歧。你很难找到一本类似《民粹主义宣言》的经典著作,系统阐述其核心教义和政治蓝图。相反,正如许多政治学者所指出的,民粹主义更准确地说是一种\textbf{政治逻辑}、一种\textbf{修辞风格},或者一种看待世界的方式。它像一个“空心”的框架,可以被不同内容所填充,可以与各种具体的意识形态相结合。

那么,这种政治逻辑的核心是什么呢?简而言之,民粹主义将社会想象成两个同质化且相互对抗的阵营:一方是“\textbf{纯洁的、统一的人民}”(the pure, unified people),另一方是“\textbf{腐败的、自私的精英}”(the corrupt, selfish elite)。民粹主义者认为,政治的本质就是一场“人民”与“精英”之间不可调和的道德斗争。“人民”被描绘成勤劳善良、拥有常识、代表着国家真正意志和核心价值的沉默大多数;而“精英”则被指责为脱离群众、道德败坏、滥用权力、只顾自身利益,甚至与外国势力勾结,出卖国家和人民。民粹主义领袖则将自己定位为“人民”的唯一真实代表,是那个能够倾听“人民”心声、为“人民”代言、带领“人民”从“精英”手中夺回权力的救世主。

这种“人民vs精英”的二元对立叙事,具有极强的道德感召力和情感动员力。它将复杂的政治、经济、社会问题简化为一场正义与邪恶的较量,为那些在快速变迁的社会中感到失落、困惑和愤怒的民众提供了一个简单明了的解释框架和情感宣泄的出口。它告诉人们:你们的困境不是你们的错,也不是因为不可避免的客观规律,而是因为有一小撮坏人窃取了本该属于你们的东西,背叛了你们的信任。

正是因为其“空心”的特性,民粹主义的逻辑可以灵活地嫁接到不同的具体议程之上,从而呈现出不同的面貌。
当它与\textbf{左翼}的议程相结合时,通常表现为\textbf{经济民粹主义}。此时,“人民”主要指那些在经济上处于弱势地位的普通劳动者、失业者或穷人,而“精英”则指向大资本家、金融寡头、跨国公司以及维护其利益的建制派政客。左翼民粹主义者承诺进行财富再分配,加强国家对经济的干预,保护本国产业和工人免受全球化和新自由主义的冲击,强调社会公平和经济正义。拉丁美洲历史上曾多次出现这类民粹主义浪潮,近年来在一些西方国家,呼吁关注贫富差距、反对紧缩政策的运动中,也带有经济民粹主义的色彩。

而当它与\textbf{右翼}的议程相结合时,则更多地表现为\textbf{民族主义民粹主义}或\textbf{文化民粹主义}。此时,“人民”通常被定义为特定民族、种族或文化群体的“土著”成员,他们被认为是国家特性和传统价值观的承载者。而“精英”则被指责为推行多元文化主义、纵容移民、向超国家机构(如欧盟)让渡主权,从而削弱了国家认同和文化纯洁性的“世界主义者”或“全球主义者”。右翼民粹主义者往往强调国家主权至上,主张限制移民,捍卫传统文化和生活方式,并对少数族裔、外来者或被视为“非我族类”的群体持排斥态度。近年来在欧美国家兴起的许多民粹主义运动,大多属于这一类型。

无论是左翼还是右翼,民粹主义者都倾向于宣称自己掌握着“人民的共同意志”(general will),并以此为由,质疑甚至否定代议制民主中的权力制衡、少数派权利、专家意见以及独立的司法和媒体监督。在他们看来,这些机制往往是“精英”用来阻挠“人民”意愿实现的工具。因此,民粹主义的兴起,不仅仅是政策偏好的转变,更可能对民主制度的运作方式和基本原则构成深远挑战。理解了民粹主义的这一核心逻辑和多变形态,我们才能更清晰地把握它在全球范围内的蔓延轨迹及其背后的深层动因。

为了系统地解开民粹主义的谜团,本书将分为四个部分,带领读者进行一次深入的探索之旅:

\begin{itemize}
    \item \textbf{第一部分:沃土——旧秩序为何崩裂。} 在这一部分,我们将深入挖掘现代社会中催生民粹主义需求的土壤。我们将审视全球化浪潮在创造巨大财富的同时,如何在一些地区和人群中留下了深深的失落感和怨恨;我们将探讨身份认同的焦虑、对传统生活方式逝去的怀旧,如何在文化层面为民粹主义的叙事提供了养分;我们还将分析主流政治的失灵——传统政党为何失去民众信任,政治精英如何变得与社会脱节,从而为民粹主义者留下了巨大的政治真空。
    \item \textbf{第二部分:剧本——全球民粹主义操盘手。} 在这里,我们将从“为何发生”转向“如何运作”以及“关键人物是谁”。我们将开启一场全球巡礼,近距离观察不同类型的民粹主义领袖及其独特的“剧本”。从美国的特朗普、英国的脱欧派,到匈牙利的欧尔班;从印度的莫迪、菲律宾的杜特尔特,再到拉丁美洲的左翼挑战者。我们将比较他们的崛起路径、动员策略、执政方式以及他们如何巧妙地利用媒体和民众情绪来巩固权力。
    \item \textbf{第三部分:后果——民粹主义时代下的生活。} 民粹主义的崛起不仅仅是政治版图的重划,它对我们的社会、我们对真相的认知乃至整个国际秩序都带来了深远的影响。本部分将分析民粹主义如何侵蚀共享的现实基础,发动针对独立媒体和专业知识的“战争”;它如何通过“我们vs他们”的对立逻辑,加剧社会分裂与政治极化;以及它内向的“本国优先”倾向,如何冲击着二战后建立的全球合作体系。
    \item \textbf{第四部分:十字路口——民主的应对与权力的未来。} 在全书的最后,我们将展望前路。民粹主义的挑战是严峻的,但并非不可逾越。我们将评估主流政治力量的应对策略,分析其成败得失;同时,我们也将聚焦来自公民社会自下而上的抵抗力量,以及民主制度本身的韧性与革新潜力。本书无意提供简单的答案或宿命的预测,而是希望通过对这场全球现象的深度剖析,为读者提供一套理解我们这个动荡时代的思想工具,帮助我们更清醒地认识权力的真实面貌,更有效地参与到塑造未来的历史进程之中。
\end{itemize}