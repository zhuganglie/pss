\part{十字路口——民主的应对与权力的未来}
\chapter{中间派能否守住阵地?建制派的反击}

\textbf{本章论点:} 主流政治力量并非被动挨打,他们也在适应和反击,但成效不一。

在本书的前三个部分,我们如同侦探一般,追寻着民粹主义这股席卷全球的政治风暴的踪迹。我们勘察了它赖以生存的“沃土”——那是被全球化遗忘的经济废墟,是身份认同焦虑的文化裂谷,更是主流政治失灵后留下的权力真空。我们描绘了全球各地“操盘手”们的“剧本”——他们如何利用民众的愤怒与恐惧,将自己塑造成人民的救世主。我们还评估了这场风暴过境后留下的“后果”——对真相的侵蚀,对社会凝聚力的撕裂,以及对国际秩序的冲击。

至此,一幅令人忧虑的图景已经展开:民粹主义者们或已登上王座,或已成为议会中不可忽视的力量,他们似乎正在改写21世纪的政治规则。那么,故事的另一方呢?那些被民粹主义者们斥为“腐败精英”、“建制派”和“旧势力”的主流政治力量,难道就真的如此不堪一击,只能坐以待毙,眼睁睁地看着自己建立的秩序分崩离析吗?

答案是否定的。政治舞台从来不是单方面的独角戏。当一股强大的力量崛起时,必然会激发出相应的反作用力。在民粹主义浪潮的冲击下,那些曾经高枕无忧的主流政党和政治家们,正从最初的错愕与震惊中清醒过来。他们被迫走出舒适区,开始审视自身的失败,并尝试用各种方式进行反击。这场“建制派的保卫战”并非总能成功,甚至常常显得笨拙和迟缓,但它确实正在发生。

这场反击战没有统一的作战手册,更像是一场在迷雾中的摸索。在不同的国家,面对不同的敌人,主流政治力量祭出了迥异的“兵法”。有的选择高筑墙垒,试图将“野蛮人”隔绝于权力核心之外;有的则选择“师夷长技以制夷”,模仿对手的招数,希望能夺回失地;还有的,则试图点燃一盏新的明灯,用一个更具吸引力的未来愿景来驱散民粹主义的阴霾。

本章,我们将深入检视这三种核心的反击策略:\textbf{“卫生隔离带”的坚壁清野、“议题吸纳”的釜底抽薪,以及“另类愿景”}的正面迎击。通过分析它们在不同战场上的成败得失,我们将探寻一个核心问题的答案:在与民粹主义的缠斗中,民主的“免疫系统”究竟是如何运作的?中间派,还能守住最后的阵地吗?

\section{策略一:“卫生隔离带”——筑起高墙的原则与困境}
在面对一个被认为是危险、具有传染性的政治力量时,最本能的反应或许就是孤立它。在政治学中,这种策略被称为“卫生隔离带”(Cordon Sanitaire)。它的核心逻辑是:主流政党之间达成共识,无论在任何情况下,都拒绝与被视为极端的民粹主义政党进行任何形式的官方合作——不与之组建联合政府,不在议会中依赖其支持,不提名其成员担任重要职位。其目的在于,通过政治上的“隔离”,阻止民粹主义政党获得执政经验和合法性,将其永远限制在“抗议者”的角色上,并向选民传递一个明确的信号:这个政党是不可接受的,是民主肌体上的“病毒”。

这种策略植根于一种深刻的道德和历史自觉,在二战后的欧洲,它曾被用来成功地遏制法西斯主义和共产主义的残余势力。而在当代,它最坚定、最著名的实践者,莫过于德国。

\subsection{案例:德国的“防火墙”——历史记忆与现实挑战}

在德国,由于纳粹的历史原罪,对任何形式的极右翼势力保持高度警惕,已经成为战后政治文化的核心共识。当右翼民粹主义政党“德国选择党”(AfD)在2015年难民危机后异军突起,并成功进入联邦议院和所有州议会时,德国主流政坛几乎是条件反射般地启动了“卫生隔离带”。

从安格拉·默克尔领导的基督教民主联盟(CDU),到其传统对手社会民主党(SPD),再到绿党和自由民主党(FDP),所有主流政党都公开承诺,绝不与AfD合作。这道“防火墙”(Brandmauer)在很长一段时间里都相当稳固。在联邦层面,它阻止了AfD参与任何执政联盟的讨论。在州一级,它导致了一些看似奇怪的政治组合。例如,为了将AfD排除在外,一些州不得不组建由中右翼、中左翼和绿党构成的“大联合政府”,意识形态差异巨大的政党为了一个共同的“敌人”而被迫携手。

“防火墙”策略的优点是显而易见的。它捍卫了德国民主的道德底线,延缓了AfD“正常化”的进程,并迫使AfD始终处于一个愤怒但无能的反对者位置,难以向选民证明自己具备执政能力。

然而,随着时间的推移,这道墙的裂缝也日益明显。尤其是在前东德地区,AfD的支持率持续攀升,常常成为当地的第一或第二大党。这就给主流政党带来了巨大的执政难题。如果严格遵守“防火墙”原则,组建一个能获得多数席位的政府就变得异常困难,有时甚至需要所有其他政党联合起来才能勉强凑够席位。这不仅导致了政府的不稳定,也让AfD的选民感到自己的选票被“无视”了。AfD可以轻易地将这种局面描绘成“旧精英们抱团取暖,联合起来对抗人民意志”的阴谋,从而进一步巩固其“反体制”的形象。

近年来,这道“防火墙”已经开始出现松动的迹象。在一些地方市镇层面,中右翼的基民盟政治人物已经开始与AfD进行项目性的合作。关于是否应该在州一级也打破禁忌的辩论,在基民盟内部愈发激烈。这暴露了“卫生隔离带”策略的内在困境:当被隔离的政党强大到一定程度时,隔离墙本身就可能成为政治僵局的根源。坚守原则的道德高地,与维持政府有效运作的现实需求之间的矛盾,变得越来越难以调和。

\subsection{案例:隔离带的崩塌——瑞典与荷兰的现实选择}

如果说德国的“防火墙”还在苦苦支撑,那么在北欧的瑞典和低地之国的荷兰,我们已经目睹了“卫生隔离带”的彻底崩塌。

在瑞典,长期以来,极右翼的瑞典民主党(Sweden Democrats)一直被所有主流政党排斥。然而,随着该党在选举中节节胜利,成为全国第二大党,这道隔离带变得摇摇欲坠。2022年大选后,中右翼的温和党为了能够上台执政,最终打破了多年的禁忌,选择与瑞典民主党达成一项历史性的协议。虽然瑞典民主党没有正式入阁,但他们成为了新右翼政府在议会中的关键支持力量,并对其政策(尤其是在移民和犯罪问题上)拥有巨大的影响力。温和党的逻辑很简单:为了实现右翼的执政议程,与魔鬼交易是唯一的选择。

荷兰的故事则更为戏剧性。反伊斯兰的民粹主义者基尔特·威尔德斯(Geert Wilders)和他的自由党(PVV)在荷兰政坛活跃了近二十年,一直被视为政治“贱民”。但在2023年的选举中,自由党出人意料地一跃成为国会第一大党。这一结果,如同一颗政治核弹,瞬间炸毁了荷兰的“卫生隔离带”。所有主流政党都被迫面对一个残酷的现实:他们无法再假装这个国家的首要政治力量不存在。组建新政府的谈判,无论多么艰难和不情愿,都必须围绕着威尔德斯展开。

瑞典和荷兰的案例揭示了一个残酷的政治法则:\textbf{“卫生隔离带”的有效性,与被隔离政党的选举实力成反比。}当一个民粹主义政党能够持续吸引20\%、25\%甚至更多的选票时,主流政党要维持对其的完全隔离,就需要付出极高的政治代价,甚至可能意味着将自己永远放逐到反对党的席位上。在权力的诱惑和现实的压力面前,曾经坚不可摧的原则,往往会变得异常脆弱。

\section{策略二:“议题吸纳”——偷走对手的雷声}
如果筑墙隔离行不通,那么另一种看似更聪明的策略便是“招安”——不是招安民粹主义者本人,而是“招安”他们的议题。这种策略被称为“议题吸纳”(Issue Co-optation),其核心逻辑是:主流政党,特别是中右翼政党,主动采纳一部分民粹主义政党的核心主张,尤其是在移民、国家安全和文化认同等议题上,用更“温和”、“体面”的语言和方式来包装它们。他们希望通过此举,向那些被民粹主义所吸引的选民证明:“你们的担忧我们听到了,你们不需要投票给那些极端分子,我们同样可以解决这些问题。”这就像是试图在雷雨天偷走对手的雷声,让其只剩下空洞的闪电。

\subsection{案例:奥地利的“神童”与丹麦的“左转”}

奥地利前总理塞巴斯蒂安·库尔茨(Sebastian Kurz)曾是“议题吸纳”策略的典范。当他接管中右翼的奥地利人民党(ÖVP)时,该党正受到极右翼的自由党(FPÖ)的严重侵蚀。库尔茨采取了大胆的行动:他几乎全盘复制了自由党在移民和伊斯兰问题上的强硬立场,用年轻、干练、现代的形象,重新包装了这些右翼民粹主义的核心议题。他承诺严格控制边境,打击“政治伊斯兰”,并强调奥地利的文化身份。

结果是惊人的。在2017年的选举中,库尔茨领导的人民党大获全胜,不仅从自由党手中抢回了大量选民,还成功地以主导伙伴的身份,与自由党组建了联合政府。他似乎完成了一次完美的“驯化”:通过采纳对手的议题,他不仅赢得了选举,还将曾经的威胁变成了自己的执政小伙伴。

然而,库尔茨的成功也揭示了这种策略的巨大风险。首先,它极大地“正常化”了极右翼的言论。当主流政党的领袖都在谈论“移民威胁”时,这些议题便不再是边缘化的禁忌,而成为了政治辩论的中心。整个国家的政治光谱,不可避免地向右移动。其次,与民粹主义者共舞,本身就充满危险。库尔茨的联合政府最终因自由党卷入的“伊维萨丑闻”而垮台,他本人后来也因腐败指控而辞职。这表明,试图利用民粹主义的火焰来取暖,一不小心就可能引火烧身。

一个更有趣的变体发生在丹麦。在这里,采取“议题吸纳”策略的,反而是中左翼的社会民主党。在长期流失工人阶级选票给右翼民粹的丹麦人民党之后,社民党领袖梅特·弗雷德里克森(Mette Frederiksen)进行了一次惊人的政策转向。她宣布,社民党将在坚持传统左翼经济政策(高福利、强工会)的同时,采纳极其严格的移民政策,包括设立“零庇护申请者”的目标。她成功地将“福利国家的捍卫者”和“国家边界的守护者”这两个形象结合起来,赢得了2019年的大选,并有效遏制了丹麦人民党的势头。

丹麦的案例似乎提供了一种新的可能性:中左翼政党是否可以通过在文化和移民议题上变得强硬,来赢回流失的传统蓝领选民?但这同样引发了激烈的争议。批评者认为,这是对左翼国际主义和人道主义价值观的背叛,为了选票而牺牲了原则。

\subsection{案例:英国保守党的“豪赌”}

英国保守党与脱欧的故事,则是“议题吸纳”策略最惊心动魄、后果也最深远的案例。在21世纪初,为了应对日益壮大的英国独立党(UKIP)的挑战,时任保守党首相戴维·卡梅伦决定采取一次终极“议题吸纳”——承诺就英国的欧盟成员国身份举行全民公投。他天真地以为,这可以一劳永逸地解决党内外的疑欧纷争,让选民用一次投票来“终结”UKIP的吸引力。

我们都知道后来的故事。这场旨在“偷走雷声”的豪赌,最终引爆了一场真正的雷暴。卡梅伦输掉了公投,也输掉了自己的政治生涯。然而,故事并未就此结束。在脱欧成为既定事实后,鲍里斯·约翰逊领导的保守党,反而将“议题吸纳”策略发挥到了极致。他不再是半推半就,而是完全、彻底地拥抱了“硬脱欧”的民粹主义叙事,承诺“完成脱欧”(Get Brexit Done)。通过这种方式,他成功地吸收了几乎全部的脱欧派选票,摧毁了残余的脱欧党(Brexit Party),并在2019年赢得了议会的压倒性多数。

英国的经历如同一部政治惊悚片,它告诉我们,“议题吸纳”是一把双刃剑。它可能导致灾难性的误判,也可能在特定条件下带来巨大的选举成功。但无论如何,它都将深刻地改变一个政党乃至一个国家的本质。当主流政党开始使用民粹主义的语言、采纳民粹主义的政策时,它可能在战术上赢得了战斗,但在战略上,民粹主义的逻辑或许已经赢得了战争。

\section{策略三:“另类愿景”——点燃一盏新的希望之灯}
如果说“卫生隔离带”是消极防御,“议题吸纳”是投机取巧,那么第三种策略则最具雄心,也最为艰难。它既不回避也不模仿,而是选择正面迎击。它试图用一个全新的、积极的、面向未来的替代愿景,来对抗民粹主义者那套基于怀旧、愤怒和恐惧的叙事。它不与对手在泥潭里缠斗,而是试图将战场拉到一片新的高地上。它要告诉选民,你们不只有“回到过去”和“维持现状”这两个选项,还有第三条路——一条通往更美好未来的道路。

\subsection{案例:2017年马克龙的“奇迹”}

2017年的法国总统大选,是“另类愿景”策略最辉煌的胜利,也是其魅力的最佳展示。当时的法国,正处在民粹主义爆发的边缘。主流的左翼社会党和右翼共和党都因丑闻和执政失败而声名狼藉,极右翼“国民阵线”的玛丽娜·勒庞势头正盛,似乎距离爱丽舍宫只有一步之遥。

就在此时,一个年仅39岁、从未参加过选举、来自建制内部却又宣称要打破建制的“局外人”——埃马纽埃尔·马克龙——横空出世。他创建了自己的政治运动“前进!”(En Marche!),并提出了一个与当时政治氛围格格不入的愿景。

当勒庞鼓吹退出欧盟、关闭边境、拥抱民族主义时,马克龙的旗帜上写着的却是:拥抱欧洲、拥抱全球化、拥抱改革。他没有试图在移民问题上比勒庞更强硬,也没有回避经济改革的阵痛。相反,他将整场选举重新定义为一场根本性的选择:不是传统的左与右之争,而是开放与封闭之争,是进步与倒退之争,是希望与恐惧之争。

他本人,就是这个愿景的化身:年轻、聪明、充满活力,说着流利的英语,对科技和未来充满信心。他承诺要打破法国沉闷的政治僵局,释放经济活力,让法国重新成为欧洲的领导者。这个愿景,对于那些厌倦了旧政客、对勒庞的排外主义感到恐惧的选民——尤其是城市里的年轻人、受过良好教育的中产阶级——具有强大的吸引力。他们看到了一个可以投票支持(for)的选项,而不仅仅是投票反对(against)勒庞。最终,马克龙以压倒性优势获胜,创造了一个政治奇迹。

然而,马克龙的成功也揭示了这种策略的脆弱性。首先,它极度依赖于领袖个人的魅力和特定的政治时机(旧政党的集体崩溃)。这种成功很难被复制。其次,宏大的愿景在执政的日常琐碎中很容易褪色。马克龙上任后,其雷厉风行的改革和高高在上的“朱庇特式”执政风格,让他很快被贴上了“富人总统”的标签,并引发了“黄背心”运动等强烈的社会反弹。他所对抗的“反精英”情绪,在他执政期间反而愈演愈烈。2022年,他虽然成功连任,但其胜利更多地被视为选民“捏着鼻子”再次阻击勒庞的结果,而非对其愿景的热情拥护。

\subsection{案例:拜登的“灵魂之战”}

2020年美国大选,乔·拜登的胜利可以被看作是“另类愿景”策略的一个变体。与马克龙那种充满未来感的革新愿景不同,拜登提供的“另类愿景”更像是一种\textbf{“回归常态”}的承诺。

面对唐纳德·特朗普四年执政带来的混乱、分裂和对民主规范的持续攻击,拜登将竞选的核心定调为一场“为国家灵魂而战”(Battle for the Soul of the Nation)。他承诺要恢复体面、同情心和团结,要重建被特朗普破坏的联盟,要让政府重新由“专家”而非“煽动家”来领导。

这是一种“以静制动”的策略。当特朗普在集会上疯狂输出时,拜登则展现出一种沉稳、可靠的“长者”形象。他提供的愿景,不是一个激动人心的新世界,而是一个人们所熟悉的、可预测的、不再每天充满戏剧性冲突的旧世界。对于那些被特朗普时代搞得筋疲力尽的美国人来说,这种对“正常”的渴望,本身就构成了一种强大的吸引力。拜登的胜利表明,在经历了民粹主义的狂热之后,对稳定和秩序的向往,也可以成为一种能够动员多数选民的“另类愿景”。

\section{结语:没有灵丹妙药,唯有漫长而艰难的跋涉}
通过检视这三种主流的“反击”策略,一幅复杂的图景浮现出来:在与民粹主义的斗争中,没有一招制敌的“银色子弹”。

“卫生隔离带”是一项高尚的原则性立场,但在民粹主义政党日益壮大的选举现实面前,它正变得越来越难以维系,甚至可能引发政治僵局,反过来成为民粹主义者攻击“体制”的口实。

“议题吸纳”是一场危险的政治赌博。它或许能在短期内赢得选票,但长期来看,它往往以牺牲自身原则、拉低整个政治辩论的底线为代价,最终可能正中民粹主义下怀。

“另类愿景”无疑是最理想、最鼓舞人心的策略,但它对领袖魅力、政治时机和选民心态的要求都极为苛刻,难以复制,且在现实的执政考验中极易磨损。

那么,建制派的出路究竟在何方?或许,真正的答案,并不在于选择哪一种战术,而在于回归政治的本质。最有效的反击,最终必须回到我们在本书第一部分所探讨的那些根源性问题上。这意味着,主流政党必须停止玩弄战术,开始做那些真正艰难的“功课”:

第一,学会倾听。 他们必须放下精英式的傲慢,真正去倾听和理解那些被全球化和文化变迁所抛下的群体的痛苦与焦虑。不能再简单地将他们斥为“无知”、“偏执”或“无可救药者”,而是要承认他们不满的合理性。

第二,拿出方案。 倾听之后,必须提出并实施能够真正解决问题的政策。这意味着要严肃对待收入不平等、产业空心化、公共服务缺失以及移民融入等挑战。这需要的是具体的、可信的、能够让民众感受到生活得到改善的行动,而不仅仅是漂亮的竞选口号。

第三,自我革新。 最重要的一点,主流政治力量必须清理自己的门户。他们必须与政治腐败、利益输送和“旋转门”现象进行彻底的切割,用行动来重建自身的信誉。他们必须向民众证明,自己并非民粹主义者口中那个只为自身利益服务的“腐败精英”,而是真正值得信托的公共利益守护者。

与民粹主义的斗争,归根结底,不是一场关于策略的智力游戏,而是一场关于民主治理能力的严峻考验。它考验的是,一个尊重多元、崇尚理性、保障权利的民主制度,是否还能比一个诉诸愤怒、煽动分裂、许诺强权的民粹主义模式,为更广大的民众提供一个更安全、更繁荣、更有尊严的未来。

王座并非注定失落,阵地也并非无法守住。但要赢得这场漫长而艰难的“保卫战”,建制派的先生们不能再指望用旧地图找到新大陆。他们必须证明,自己依然配得上人民的信任,依然有能力领导国家走出当前的困境。否则,无论他们筑起多高的墙,或是模仿多逼真的雷声,都无法阻止民众将希望投向那些承诺要“将权力还给人民”的挑战者们。而公民社会自下而上的抵抗力量,又将在这场博弈中扮演怎样的角色?这正是我们下一章将要探讨的议题。