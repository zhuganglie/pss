\chapter{非自由民主派:欧尔班的匈牙利}

在民粹主义的万花筒中,如果说特朗普代表着一种喧嚣、冲动、常常在边缘试探的右翼民族主义,那么维克多·欧尔班(Viktor Orbán)则呈现出截然不同的面貌:他是一位冷静、精明、极具战略眼光的“非自由民主派”建筑师。他没有选择推翻民主,而是选择掏空它,将其内核替换为一种看似民主实则威权的统治模式。匈牙利,这个中欧小国,在欧尔班的领导下,成为了全球政治实验室中一个令人不安的样本,展示了民粹主义者如何在赢得选举后,系统性地“捕获”国家机器,从而实现长期的权力垄断。

本章将深入探讨欧尔班的“剧本”,揭示他如何一步步将匈牙利从一个新兴的自由民主国家,改造为一个“非自由民主”的堡垒。我们将看到,这并非一蹴而就的革命,而是一场耐心而周密的“长线游戏”,其每一步都经过深思熟虑。通过与特朗普的对比,我们也将理解为何欧尔班的模式,对那些渴望在民主外壳下巩固自身权力的潜在威权领袖,具有如此巨大的吸引力。

\section{从自由斗士到非自由建筑师:欧尔班的蜕变}

要理解欧尔班的“非自由民主”模式,我们首先需要回顾他独特的政治轨迹。维克多·欧尔班并非一开始就是民粹主义者。在1989年东欧剧变时,他以一位年轻、充满激情的自由主义学生领袖形象登上历史舞台。在匈牙利共产党政权摇摇欲坠之际,正是他,在民族英雄伊姆雷·纳吉的重新安葬仪式上,公开呼吁苏联军队撤出匈牙利,并要求举行自由选举。那时的他,是西方自由民主价值观的坚定拥护者,是匈牙利向欧洲回归的象征。

然而,政治的舞台充满了变数。在1990年代,欧尔班领导的青年民主主义者联盟(Fidesz)最初是一个自由主义政党。但随着时间的推移,他敏锐地察觉到匈牙利社会深层的不满和渴望。在经历了一段在野时期后,欧尔班和他的青民盟开始向右转,拥抱民族主义、保守主义和基督教价值观。他不再是那个呼吁自由的青年,而是一个深谙民意、善于利用社会情绪的政治家。

2010年,在匈牙利经历了一系列经济危机和左翼政府的腐败丑闻之后,欧尔班和青民盟以压倒性优势赢得了议会选举,获得了三分之二的绝对多数席位。这一胜利,不仅是选民对旧秩序的厌倦,更是欧尔班“非自由民主”实验的开端。他不再满足于仅仅执政,而是要彻底重塑匈牙利的政治和社会结构。

\section{系统性“捕获”国家机器:欧尔班的权力巩固剧本}

欧尔班的“剧本”并非一味地煽动民意,而是以一种外科手术般的精准,对匈牙利的民主机构进行系统性改造。他深知,仅仅赢得选举是不够的,真正的权力在于对国家机器的绝对控制。

\subsection{修改宪法:重塑国家基石}

获得三分之二多数席位后,欧尔班做的第一件事就是修改宪法。这并非简单的修修补补,而是一次彻底的“宪法革命”。2011年,匈牙利通过了新宪法,取代了1989年以来的旧宪法。新宪法不仅将“上帝与基督教”写入序言,强调民族认同,更重要的是,它为欧尔班政府巩固权力提供了法律依据。

新宪法缩减了宪法法院的权力,限制了其对议会立法的审查能力。它还允许议会通过“基本法”,这些法律需要三分之二多数才能修改,从而将青民盟的政策永久化,即使未来失去多数席位,也难以被推翻。例如,关于家庭、媒体、司法等领域的关键法律,都被提升到基本法的地位。这就像在民主的棋盘上,欧尔班不仅赢得了当前这局棋,还修改了棋盘的规则,让未来的对手难以翻盘。

\subsection{安插亲信,掌控法院:司法独立之殇}

司法独立是民主制度的基石,它确保法律面前人人平等,并对政府权力形成制衡。欧尔班深知这一点,因此将司法系统作为其“捕获”战略的重点。

他首先通过立法降低了法官的强制退休年龄,导致数百名经验丰富的法官被迫退休。这些空缺随即被欧尔班政府任命的亲信填补。此外,他还设立了新的司法机构,如国家司法办公室,并任命忠于青民盟的人士担任负责人,从而将司法行政权牢牢掌握在政府手中。

宪法法院也未能幸免。通过修改宪法和任命新法官,宪法法院的组成发生了根本性变化,其判决往往与政府的意愿高度一致。曾经作为民主守护者的法院,逐渐沦为政府政策的橡皮图章。当公民或反对派试图通过法律途径挑战政府时,他们发现司法的大门已经不再敞开。

\subsection{控制媒体:塑造“我的事实”}

在信息时代,媒体是塑造公众舆论、监督政府权力的重要力量。欧尔班政府对媒体的控制,是其“非自由民主”模式中最显著的特征之一。

首先,公共媒体(如国家电视台和电台)被完全置于政府控制之下。其新闻报道高度偏向政府,对反对派进行抹黑,并对政府的批评声音进行过滤。这些媒体成为了政府的宣传机器,而非独立的公共服务机构。

其次,私人媒体也未能幸免。通过政府支持的商业集团收购、广告投放倾斜、以及对独立媒体的打压(如撤销广播执照、施加财政压力),大量私人媒体被亲政府势力控制。一个庞大的亲政府媒体帝国逐渐形成,涵盖了报纸、电视台、广播电台和网络媒体。独立媒体的空间被急剧压缩,记者面临着巨大的压力,甚至人身安全威胁。

匈牙利还设立了媒体委员会,其成员由政府任命,拥有对媒体内容进行审查和罚款的权力。这使得媒体在报道时不得不自我审查,以避免触怒当局。最终,匈牙利形成了一个高度同质化的媒体环境,公众获取信息的渠道被严重限制,只能听到政府希望他们听到的声音。这为政府推行其议程、打击异见创造了有利条件。

\subsection{污名化公民社会:压制异见}

公民社会组织(NGOs)、大学和智库是民主社会中重要的制衡力量,它们代表着不同的利益和观点,并对政府进行监督。欧尔班政府将这些组织视为潜在的威胁,并采取措施对其进行压制和污名化。

最著名的例子是针对由美国金融家乔治·索罗斯资助的非政府组织的打击。欧尔班政府将索罗斯描绘成一个“全球主义者”和“外国代理人”,指责他试图通过资助NGOs来干涉匈牙利内政,破坏国家主权。2017年,匈牙利通过了《外国资助组织透明法》,要求接受外国资助的NGOs进行注册并公开其资金来源,否则将面临罚款甚至关闭。这一法律被批评为旨在恐吓和压制异见。

大学也成为攻击目标。中欧大学(CEU),一所由索罗斯创办的国际知名大学,因其自由主义的学术氛围和对政府的批评,长期受到欧尔班政府的打压。最终,在政府的持续压力下,中欧大学被迫将其大部分课程从布达佩斯迁至维也纳。这不仅是对学术自由的侵犯,更是向所有独立机构发出的警告:不与政府保持一致,就将面临生存危机。

\subsection{操纵选举制度:确保长期执政}

尽管欧尔班政府声称自己是民主选举的产物,但它也对选举制度进行了修改,以确保青民盟的长期执政。

通过重新划分选区(“杰利蝾螈”),青民盟将支持自己的选民集中在某些选区,同时将反对派选民分散到多个选区,从而在选票数量不占绝对优势的情况下,最大化其在议会中的席位。此外,选举法也进行了修改,例如增加了政党进入议会的门槛,使得小党派更难获得席位,从而有利于大党(青民盟)。

这些措施使得匈牙利的选举虽然形式上依然存在,但其公平性和竞争性大打折扣。反对派发现,即使他们能够团结起来,也难以在被操纵的选举制度下挑战青民盟的霸权。

\section{国际舞台上的“孔雀开屏”:对内宣传与对外挑战}

欧尔班的“非自由民主”模式不仅在国内大行其道,在国际舞台上,他也有一套独特的“孔雀开屏”式外交。他深知,匈牙利作为欧盟成员国,不能完全脱离国际社会。因此,他的国际策略旨在服务于其国内的权力巩固,同时挑战自由主义的国际秩序。

欧尔班经常在欧盟峰会、联合国大会等国际场合,扮演一个“民族主权捍卫者”的角色。他公开批评欧盟的“官僚主义”和“全球主义精英”,反对欧盟的移民政策,并宣称自己是在捍卫匈牙利的国家利益和基督教文明。他将欧盟的批评描绘成外部势力对匈牙利主权的干涉,从而在国内激起民族主义情绪,巩固自己的支持率。

这种外交策略的本质是“表演给国内观众看”。当欧盟对匈牙利的民主倒退表示担忧时,欧尔班会将其解读为“布鲁塞尔”对匈牙利的打压,从而强化他在国内的“反抗者”形象。他利用国际舞台来强化其在国内的叙事:即他是一个敢于对抗强大外部势力的民族英雄。

与此同时,欧尔班也积极寻求与其他持相似观点的国家建立联系,例如波兰的法律与公正党政府,以及一些中东欧国家的保守派领导人。他试图构建一个“非自由民主”的国际联盟,共同对抗他所称的“自由主义霸权”。这种策略不仅为他提供了国际上的支持,也为其他潜在的威权领袖提供了效仿的榜样。

\section{有耐心的民粹主义者:欧尔班与特朗普的对比}

将欧尔班与特朗普进行对比,能够更清晰地揭示“非自由民主”模式的独特之处,以及它为何对其他潜在的威权领袖具有如此大的吸引力。

\subsection{战略与混乱:两种截然不同的风格}

特朗普的民粹主义是喧嚣的、冲动的、常常是混乱的。他的权力巩固尝试往往是即兴的、反应性的,并且常常在法律和制度的边缘游走。他试图通过推特、大型集会和直接的言语攻击来绕过传统媒体和机构,但他的努力常常因为缺乏系统性规划和内部阻力而受挫。例如,他试图挑战2020年大选结果,但最终被法院、选举官员和国会所阻止。他的风格更像是一场持续的政治风暴,虽然声势浩大,但缺乏精准的打击。

相比之下,欧尔班的民粹主义是冷静的、精明的、极具战略性的。他不是一个“推特治国”的领导人,而是一个深思熟虑的建筑师。他的每一步行动,从修宪到控制媒体,从打击NGO到重塑司法,都是经过精心策划和系统性实施的。他没有试图推翻民主,而是选择在民主的框架内,通过合法(但非自由)的手段,一点点地侵蚀和掏空民主的内核。他像一个耐心的外科医生,精确地切除民主的制衡机制,而不是像一个狂暴的破坏者。

\subsection{制度的韧性与制度的捕获}

特朗普的美国,尽管经历了民粹主义的冲击,但其强大的制度韧性(如独立的司法、自由的媒体、成熟的官僚体系)在很大程度上抵御了他的权力扩张。美国的制衡机制,虽然受到挑战,但最终发挥了作用。

而在匈牙利,欧尔班则成功地“捕获”了这些制度。他没有试图推翻法院,而是通过任命亲信来控制它;他没有废除媒体,而是通过收购和审查来控制它;他没有解散议会,而是通过修改选举法来确保其多数地位。他将民主的“外壳”保留下来,但掏空了其“内核”,使其不再能有效制衡权力。这种“制度捕获”使得他的权力巩固更加持久和难以逆转。

\subsection{为何欧尔班模式更具吸引力?}

欧尔班的模式之所以对其他潜在的威权领袖具有巨大的吸引力,原因有以下几点:

合法性外衣: 它提供了一种在民主框架内实现威权统治的“合法”路径。领导人可以通过赢得选举来上台,然后逐步侵蚀民主,而无需诉诸军事政变或暴力镇压。这使得其政权在国际上更具迷惑性,也更难被直接谴责为独裁。
规避国际制裁: 相比于赤裸裸的独裁,这种“非自由民主”模式在国际上更容易规避严厉的制裁和谴责。虽然欧盟对匈牙利启动了“法治程序”,但由于欧盟内部的复杂性,这些程序往往进展缓慢,难以产生决定性影响。
持久性: 通过系统性地改造国家机器,这种模式能够确保领导人长期执政,并为未来的权力继承做好准备。它不像特朗普那样,权力高度依赖于个人魅力和情绪煽动,一旦个人离开,其影响力可能迅速消退。
蓝图效应: 欧尔班的匈牙利为其他国家提供了一个清晰的“剧本”或“操作手册”,展示了如何一步步地削弱民主制衡,巩固个人权力。这种模式的成功,无疑会鼓励其他国家的领导人效仿。

\section{非自由民主的未来:对全球的警示}

欧尔班的匈牙利,是21世纪民主面临的新挑战的缩影。它不再是简单的军事政变或共产主义独裁,而是一种更具欺骗性和渗透性的权力形式。它利用民主的工具来摧毁民主,利用民意来压制异见,利用法律来巩固非法治的统治。

这种“非自由民主”模式的兴起,对全球民主构成了严峻的警示。它提醒我们,民主并非一劳永逸的成就,而是一个需要持续捍卫和维护的脆弱体系。当公民社会被压制,媒体被控制,司法不再独立,即使选举依然存在,其民主的实质也已荡然无存。

欧尔班的匈牙利,是一个活生生的案例,它告诉我们:民粹主义者一旦掌握权力,他们不仅会挑战旧秩序,更会系统性地重塑国家,以确保其权力的永久化。理解欧尔班的“剧本”,对于我们识别和应对全球范围内日益增长的“非自由民主”趋势,至关重要。它迫使我们思考,当选票不再是民主的唯一衡量标准时,我们该如何定义和捍卫真正的自由与公正。